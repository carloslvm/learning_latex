\documentclass [a4paper, 11pt]{article}

\begin{document}

\title{Another \LaTeX\ exercise: What is Shell Scripting?}
\author{Carlos}
\date{July 14, 2019}
\maketitle

\section{Introduction}
In this pdf document created with \LaTeX\ , we learn what
shell scripting is and what it is used for at a basic
level.
\\ % I want the next line to be indented.

The text that you are about to read was extracted from the
official Linux documentation in order to keep practicing
typesetting with \LaTeX\ .

\section{Shell Scripting}
A shell script is a quick-and-dirty method of prototyping
a complex application. Getting even a limited subset of
the functionality to work in a script is often a useful
first stage in project development. In this way, the
structure of the application can be tested and tinkered with,
and the major pitfalls found before proceeding to the final
coding \textit{\textbf{C, C++, Java, Perl or Python}}.
\\ % Once again, I want the next line to be indented.

Shell scripting hearkens back to the classic UNIX philosophy
of breaking complex projects into simpler subtasks, of chaining
together components and utilities. Many consider this a better,
or at least esthetically pleasing approach to problem solving
than using one of the new generation of high-powered 
all-in-one languages, such as \emph{Perl}, which attempt to be
all things to all people, but at the cost of forcing you to
alter your thinking processes to fit the tool.\\
% This time I do not care whether or not the next line is indented.
This text was taken from \emph{http://tldp.org/LDP/abs/html/why-shell.html}.

\end{document}
