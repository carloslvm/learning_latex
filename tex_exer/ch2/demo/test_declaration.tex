\documentclass[a4paper,11pt]{article}

\begin{document}
\title{Testing sffamily, ttfamily, rmfamily.}
\author{Carlos Valdez}
\date{June 24, 2019}
\maketitle

\section{Introduction}

In this document I use some basic \LaTeX\ declarations that affect
the whole text when \emph{they are declare at least once.}\\

These declarations are: \emph{\textbf{sffamily, ttfamily, rmfamily.}}\\

The following text was taken from \texttt{www.openvpn.net}.

\section{OpenVpn (text sample)}

% This paragraph should be printed in san serif font.
\sffamily \textbf{OpenVPn} is a full featured open source SSL VPN solution that
accommodates a wide range of configurations, including remote access,
site to site VPNs. Wi-Fi security, and enterprise scale remote access
solutions with load balancing fallover and fine grained access controls.
\\

% This paragraph should be printed in typewriter font.
\ttfamily \textbf{OpenVpn} offers a cost effective lightweight alternative
to other VPN technologies making it the best solution for small to medium
enterprises.

\section{Conclusion}

% Return to roman font (default font)
\rmfamily Two paragraph were printed with two different fonts and here
I return to the default font.\\

The difference between commands and declarations in \LaTeX\ are that
commands need arguments in braces and only affect the text in braces,
but declarations do not use braces and affect the text from the
point you use them onwards.

\end{document}

% Summarizing declarations:

% \rmfamily		Roman family.
% \sffamily		Sans-serif family
% \ttfamily		Typewriter family
% \bfseries		Bold-face
% \mdseries		Medium
% \itshape		Italic shope
% \slshape		Slanted shape
% \scshape		Small Caps Shape
% \upshape		Upright shape
% \normalfont	Default font
% \em			Emphasize
