\documentclass[a4paper,11pt]{article}

\begin{document}

\title{Latex Lab 1}
\author{Carlos}
\date{June 2, 2019}
\maketitle

\section{Introduction}
In this lab, I put into practice the basic commands of
Latex learned in chapter 1 and the beginning of chapter 2.
\\ %Line break. To indent the next paragraph, I leave an empty line in source code.

The text I use to do this lab, was taken from sites
randomly in order to make use of the acquired knowledge of
Latex.

\section{Gutenberg Welcome}
\textbf{Project Gutenberg} offers over 59000 free ebooks.
Choose among free epub and Kindle eBooks, download them
or read them online. You will find the world's great
literature here, with focus on older works for which U.S.
copyright has expired. Thousands of volunteers digitized
and dilifently proofread the eBooks, for enjoyment and
education.
\\

This text was taken from \emph{www.gutenberg.org}

\section{Debian Operating System}
Debian is a very successful operating system, which is
pervasive in our digital lives more than people often 
imagine or are aware of. A few data points will suffice
to make this clear. At the time of writing Debian is the
most popular GNU/Linux variant among web servers:
according to \textit{\textbf{W3Techs}}, more than 10\% of
the web is Debian-powered. Think about it: how many web
sites would have you missed today without Debian? Onto
fascinating deployments, Debian is the operating system
of choice in the International Space Station. Have you
been following the work of ISS astronauts, maybe via the
social network presence of \textsl{NASA} or other
international organizations? Both the work in itself and
the posts about it have been made possible by Debian.
\\

Taken from \emph{https://debian-handbook.info/browse/stable/preface.html}

\end{document}
